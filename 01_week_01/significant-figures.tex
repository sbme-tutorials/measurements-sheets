\section*{Rationale}

\begin{itemize}

\item The number of significant figures \textit{is the number of digits between and including the least
and \emph{the most significant digits}}. 
\item The leftmost nonzero digit is called \emph{the most significant digit}; the rightmost nonzero digit, \emph{the least significant digit}.
\item If there is a decimal point in the number, then the rightmost digit is \emph{the least significant digit} even if it is a zero. 
\item These rules imply that the following
numbers have five significant figures:
\begin{itemize}
\item 1.0000
\item 2734.2
\item 53 267.
\item 428 970
\item 10 101
\item 0.008 976 0
\end{itemize} 

\end{itemize}

\subsection*{Round-off}

\begin{enumerate}
\item To round off a number, the number first is truncated to its desired length. Then the excess digits are expressed as a decimal fraction.
\item If it is greater than 1/2, we round up the least significant digit by one; if it is
less than 1/2, it is left alone. If the fraction equals 1/2, the least significant
digit is rounded up by one only if that digit is odd (Patrick F. Dunn).
\item Do not round-off intermediate results. Only the final results to be rounded-off.
\item The number of significant figures in a computed result equals the minimum number of significant figures in any number used in the computation.
\end{enumerate}


\subsection*{Scientific Notation}

\begin{itemize}
\item What happens when no decimal point is present, a zero is the rightmost digit and it is significant? This situation is ambiguous and can be avoided by expressing the number in scientific notation, where $428 970$ becomes $4.289 70 \times 10^5$ .
\item All of the digits present in scientific notation are significant. 
\end{itemize}

\begin{question}
Round off the following numbers to three significant figures: $23 421$,
$16.024$, $273.61$, $5.6850 \times 10^3$ , and $5.6750 \times 10^3$ .
\examspace{5em}

\end{question}
\begin{solution}
The answers are $23 400$, $16.0$, $274$, $5.68 \times 10^3$ , and $5.68 \times 10^3$ . Note that
$16.024$ when rounded off to three significant figures is $16.0$, where the 0 is significant
because it is to the right of the decimal point. Also note that the last two numbers
when rounded off to three significant figures become the same. This is because of our
rule for round-off when the truncated fraction equals 1/2.

\end{solution}


\begin{question}
How many significant figures does the number 001 001.0110 have?
\examspace{5em}

\end{question}
\begin{solution}
8. The left-most 2 zeros are not significant. However, the right-most zero is
significant because of the presence of a decimal point.

\end{solution}


\begin{question}
Determine in the appropriate SI units the value with the correct number
of significant figures of the work done by a $1.460 \times 10^6$ lbf force over a $2.3476$ m
\examspace{5em}

\end{question}
\begin{solution}
There are four significant figures for the force and five for the distance.
Because work is the product of force and distance (assuming that the force is applied
along the direction of motion), work will have four significant figures. The SI unit of
work is the joule, where $J = N \cdot m$. Now $1.460 \times 10^6$ lbf equals $6.495 \times 10^6 N$. So, the
work is $1.525 \times 10^6 J$ or $1.525 MJ$.

\end{solution}