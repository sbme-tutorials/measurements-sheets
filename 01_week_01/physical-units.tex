
\begin{question}[subtitle=Patrick F. Dunn Sol. 2.4]
Which of the following is not equivalent to the SI unit of energy?

\begin{enumerate}
\item ${\rm kg \cdot m^2 /s^2}$
\item ${\rm Pa \cdot m^2}$
\item ${\rm N \cdot m}$
\item ${\rm W \cdot s } $
\end{enumerate}
\examspace*{5em}

\end{question}
\begin{solution}
${\rm Pa \cdot m^2}$ \\
${\rm Pa = N/m^2}$, which implies that ${\rm Pa \cdot m^2 = N}$. A newton is a unit of force, not of energy.

\end{solution}


\begin{question}[subtitle=Patrick F. Dunn Sol. 2.7]
An astronaut weighs 164 pounds on Earth (assume that Technical English is spoken). What is the astronaut’s weight (in the appropriate SI unit and
with the correct number of significant figures) on the surface of Mars where the gravitational acceleration is 12.2 feet per second squared?
\examspace*{5em}

\end{question}
\begin{solution}
276 N \\
If weight is known in one gravitational field, then the weight of the same
mass in another gravitational field can be determined simply by multiplying by the ratio of gravitational accelerations. So, multiply the weight by 12.2/32.174 to the weight on Mars. Then multiply by 4.448 N/lbf.

\end{solution}

\begin{question}[subtitle=Patrick F. Dunn Sol. 2.8]
An astronaut weighs 162 pounds on Earth (assume that Technical En-
glish is spoken). What is the astronaut’s mass (expressed in the appropriate SI unit and with the correct number of significant figures) on the surface of Mars, where the gravitational acceleration is 12.2 feet per second squared?
\examspace*{5em}

\end{question}
\begin{solution}
73.5 kg \\
There are three significant figures. To determine the mass, first convert
the weight to mass by dividing by 32.174 to get slugs. Then use the conversion 14.594 kg = 1 slug.

\end{solution}

\begin{question}[subtitle=Patrick F. Dunn Sol. 2.10]
A robotic manipulator weighs 393 lbf (Technical English) on Earth.
What is the weight of the probe on the moon’s surface in newtons (to the nearest tenth of a newton) if the lunar gravitational acceleration is 1/6 of that on Earth?
\examspace*{5em}

\end{question}
\begin{solution}
291 N \\ 
To arrive at the probe weight on the moon, multiply the weight on earth
by 1/6. Convert to newtons by multiplying by 4.448 N/lbf.


\end{solution}



\begin{question}[subtitle=Patrick F. Dunn Sol. 2.11]
How much work is required to raise a 50 g ball 23 in. vertically upward?
Express your answer in units of ft-lbf to the nearest one-thousandth of a ft-lbf.
\examspace*{5em}

\end{question}
\begin{solution}
0.211 ft-lbf \\ 
Work is force times distance. Mass and distance are given. Force is mass
times acceleration. So, the answer is \\
${\rm mass(g) \times \frac{1 kg}{1000 g} \times 9.806650 m / s^2 \times distance(in.) \times \frac{1 ft}{12 in.} \times \frac{1 m }{3.2808 ft} \times \frac{0.737562 ft-lbf}{ N \cdot m}}$


\end{solution}

