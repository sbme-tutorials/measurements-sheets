\documentclass[a4paper,11pt,dvipsnames]{book}
\renewcommand{\familydefault}{\sfdefault}

\usepackage{standalone}
\usepackage[english]{babel}
\usepackage[top=3cm]{geometry}
\usepackage{float}
\usepackage{tabularx}
\usepackage{multirow}
\usepackage{booktabs}
\usepackage{pgfplots}
\usepackage{amsmath}
\usepackage{amssymb}
\usepackage{amsfonts}
\usepackage{siunitx}
\usepackage{tikz}
\usepackage{graphics} % for pdf, bitmapped graphics files
\usepackage{graphicx}
\usepackage{exsheets}
\usepackage{algorithm}
\usepackage{algorithmicx}
\usepackage[noend]{algpseudocode}
\usepackage{hyperref}
\usepackage{enumitem}
\usepackage{filecontents}
\usepackage{multirow}
%\usepackage{showframe}% to show frames
%\ifCLASSOPTIONcompsoc
\usepackage[caption=false, font=normalsize, labelfont=sf, textfont=sf]{subfig}
%\else
%\usepackage[caption=false, font=footnotesize]{subfig}
%\fi    

\usetikzlibrary{patterns,arrows,arrows.meta,calc,intersections,shapes,positioning,decorations.pathreplacing,decorations.markings,decorations.pathmorphing}
\usepackage{multicol}

\sisetup{output-decimal-marker={,},exponent-product=\cdot}

\DeclareSIUnit\atm{atm}
\DeclareSIUnit\dioptre{D}



\def\BState{\State\hskip-\ALG@thistlm}


\definecolor{TitleColor}{rgb}{0.65,0.04,0.07}
\definecolor{NumberColor}{rgb}{0.02,0.04,0.48}

\DeclareInstance{exsheets-heading}{fancy}{default}{
toc-reversed = true ,
indent-first = true ,
vscale = 2 ,
pre-code = \IfInsideQuestionT{\rule{\linewidth}{1pt}} ,
post-code =\IfInsideQuestionT{\rule{\linewidth}{1pt}} ,
subtitle-format = \large\scshape\color{rgb:red,0.65;green,0.04;blue,0.07} ,
number-format = \large\bfseries\color{rgb:red,0.02;green,0.04;blue,0.48} ,
points-format = \itshape ,
join = { number[r,B]title[l,B](.333em,0pt);
title[r,B]subtitle[l,B](1em,0pt)
} ,
attach =
{
main[hc,vc]number[hc,vc](0pt,0pt) ;
main[l,vc]subtitle[hc,vc](\marginparsep,0pt)
}
}



\DeclareInstance{exsheets-heading}{block-subtitle}{default}{
vscale = 2 ,
pre-code = \rule{\linewidth}{1pt} ,
post-code = \rule{\linewidth}{1pt} ,%title-format = \large\scshape\color{TitleColor} ,
number-format = \large\bfseries\color{rgb:red,0.02;green,0.04;blue,0.48} ,
subtitle-format = \large\scshape\color{black} ,
join = {
title[r,B]number[l,B](.333em,0pt) ;
title[r,B]subtitle[l,B](1em,0pt)
} ,
attach = {
main[l,vc]title[l,vc](0pt,0pt) ;
main[r,vc]points[l,vc](\marginparsep,0pt)
},
}

\DeclareQuestionClass{textbook}{textbooks}

\SetupExSheets{
  headings = fancy,
  question/print = true ,
  solution/print = false }
 % counter-format = se.qu ,
%  counter-within = section ,
  %question/pre-hook = \rule{\textwidth}{1pt},


\hypersetup{
	colorlinks = true, 
	breaklinks = true, 
	bookmarks = true,
	bookmarksnumbered = true,
	urlcolor = blue, 
	linkcolor = blue, 
	citecolor=blue,
	linktoc=page, 
	pdftitle={}, 
	pdfauthor={\textcopyright Author}, 
	pdfsubject={}, 
	pdfkeywords={}, 
	pdfcreator={pdfLaTeX}, % PDF Creator
	pdfproducer={IEEE} }





\tikzset{point/.style={circle,fill,black!80,inner sep=0pt,minimum size=#1,opacity=0.9}}
\tikzset{point/.default=3pt}\tikzset{vector/.style={line width=1pt,postaction={decorate,decoration={markings,mark=at position 1 with {\arrow{latex}}}}}}
\tikzset{block/.style={rectangle,fill=black!30,draw,minimum size=#1,opacity=0.9,align=center}}
\tikzset{block/.default=15pt}\tikzset{ball/.style={circle,fill=black!30,draw,minimum size=#1,opacity=0.9}}
\tikzset{ball/.default=5pt}\tikzset{pulley/.style={draw=black,line width=0.2pt,circle,minimum size=#1,inner sep=0pt,fill=black!10}}
\tikzset{pulley/.default=20pt}\tikzset{rod/.style={line width=2pt}}
\tikzset{rope/.style={line width=1pt}}
\tikzset{spring/.style={decorate,decoration={coil,amplitude=5pt,segment length=#1,aspect=0.3}}}
\tikzset{spring/.default=5pt}\tikzset{wall/.style={black!10,pattern=north east lines,opacity=0.3}}
\tikzset{ray/.style={line width=0.8pt,postaction={decorate,decoration={markings,mark=at position 0.5 with {\arrow{>}}}}}}
\tikzset{arrow/.style={-latex}}
\tikzset{object/.style={line width=1pt,orange,-latex}}
\tikzset{image/.style={line width=1pt,blue,-latex}}
\tikzset{doublearrow/.style={<->,>=latex,thick}}
\tikzset{brace/.style={decorate,decoration={brace,amplitude=#1}}}
\tikzset{brace/.default=5pt}




\graphicspath{{images/}} 




\makeatletter
\@addtoreset{question}{section}
\makeatother


\begin{document}
\author{Dr. Muhammed Rushdi \and Asem Alaa}

\title{Measurements and Instrumentation [SBE206A] (Fall 2018)\\ Tutorial 9}

\maketitle


\chapter*{Descriptive Statistics for Measurements Errors}


\section*{Drawing Histogram}
\begin{enumerate}
\item Count $n$ as the number of data points.
\item Determine the number of histogram bins using Sturges' rule: $k = \lceil \log_2 n \rceil+ 1$.
\item Determine the range of data points: \\$R=max\left\lbrace x_i \right\rbrace - min\left\lbrace x_i \right\rbrace$
\item Determine the width $w$ of each bin: $w = \frac{R}{k}$. You may slightly adjust the width to avoid ambiguity.
\item Pinpoint the bins boundary on the graph. To span the whole range using $k$ bins interspaced by $w$.
\item Calculate the normalized frequency of each histogram $p_i = \Sigma_i^k \frac{m_i}{n}$; where $m_i$ is the number of data points lying in the interval (i.e bin) $i$.
\item Complete the histogram after determining the height of each bin to be proportional to its normalized frequency.
\end{enumerate}

\section*{Testing the Normality of your Data}

Suppose that your data (consists of 100 points,$\mu = 499.5$, $\sigma = 8.389$) the first step (which is given) is to plot the histogram of your data. In this demo, the histogram summary is directly given to you as following: 

\begin{tabular}{|c|c|}
\hline 
Interval & $m_i$ \\ 
\hline 
$\left[ 479.5, 484.5 \right]$ & 5 \\ 
\hline 
$\left[ 484.5, 489.5 \right]$ & 8 \\ 
\hline 
$\left[ 489.5, 494.5 \right]$ & 13 \\ 
\hline 
$\left[ 494.5, 499.5 \right]$ & 23 \\ 
\hline 
$\left[ 499.5, 504.5 \right]$ & 24 \\ 
\hline 
$\left[ 504.5,509.5 \right]$ & 14 \\ 
\hline 
$\left[ 509.5, 514.5 \right]$ & 9 \\ 
\hline 
$\left[  514.5, 519.5 \right]$ & 4 \\ 
\hline 
\end{tabular} \\


The next steps is to map the histogram boundaries to the standardized normal distribution:
\begin{itemize}
\item If your data points $n>31$, use the standardized normal distribution.
\item If your data points $n \leq 31$, use the student distribution.
\end{itemize} 

\hspace{-2cm}
\begin{tabular}{|c|c|c|c|c|c|c|c|c|c|}
\hline 
Boundary & 479.5 & 484.5 & 489.5 & 494.5 & 499.5 & 504.5 & 509.5  & 514.5 &  519.5 \\ 
\hline 
$Z(\frac{x-\mu}{\sigma})$ &  -2.38 & -1.792 & -1.195 & -0.6  & -0.004 & 0.592 &  1.188 & 1.784 & 2.381 \\
\hline 
$F(z)$ & 0.0087 & 0.037 & 0.116 &  0.274 & 0.498 & 0.723 &  0.883  & 0.963 & 0.991 \\
\hline 
Interval area & & & & & & & & & \\
\hline
expected count $m'_i = np$ & & & & & & & & & \\ 
\hline
\end{tabular} \\

Finally, 
\begin{enumerate}
\item compute the $\chi^2$ as following:
\begin{equation}
\chi^2 = \Sigma \frac{(m'_i - m_i)^2}{m'_i}
\end{equation}
\item using $K=n-2=8$ as the degree of freedom, compare the obtained $\chi^2$ with the values at confidence level 90\% and 95\%.
\end{enumerate}


\section*{Uncertainty as an Estimated Variance}

Defining the squared uncertainty $u_i^2$ as
an estimate of the variance $\sigma_i^2$: 
\begin{equation}
u_c^2 \approx u_{x1}^2\left(\frac{\partial r}{\partial x_1}\right)^2 +  u_{x2}^2\left(\frac{\partial r}{\partial x_2}\right)^2 + 2u_{x1x2}\left(\frac{\partial r}{\partial x_1}\right)\left(\frac{\partial r}{\partial x_2}\right) + \cdots
\end{equation}

\section*{Single-Measurement Measurand Experiment}

\textbf{Design-stage} uncertainty is expressed as a function of the zero-order uncertainty of the instrument, $u_0$ , 
and the instrument uncertainty, $u_I$ , as:

\begin{equation}
u_d = \sqrt{u_I^2 + u_0^2}
\end{equation}

The \textbf{resolution} of an instrument is the \emph{smallest physically indicated division that the instrument displays or is marked}. The zero-order uncertainty of the instrument, $u_0$ , \textbf{is set arbitrarily to be equal to one-half the resolution}. \\

The instrument uncertainty, $u_I$, usually is stated by the manufacturer and
results from a number of possible elemental instrument uncertainties, $e_i$ .
Examples of $e_i$ are hysteresis, linearity, sensitivity, zero-shift, repeatability,
stability, and thermal-drift errors. Thus,
\begin{equation}
u_I=\sqrt{\Sigma_i^N (e_i)^2}
\end{equation}


\chapter*{Problems}


\begin{question}
Problem Statement: Some car rental agencies use an onboard global positioning
system (GPS) to track an automobile. Assume that a typical GPS’s precision is 2\%
and its accuracy is 5\%. Determine the combined standard uncertainty in position indication that the agency would have if \begin{enumerate}
\item  it uses the GPS system as is, and 
\item it recalibrates the GPS to within an accuracy of 1\%.
\end{enumerate}

\end{question}
\begin{solution}
\examspace*{10em}


\end{solution}


\begin{question}
Since you cannot measure the kinetic energy (KE) of a motorcycle directly, you settle for measuring its mass (G) and velocity (H). You determined that the average values are $\bar{m} = 500$ kg and $\bar{v} = 20$ m/s. 
Knowing that 
\begin{equation}
\text{KE} = \frac{1}{2} m v^2
\end{equation}
what is the most probable uncertainty in your computations of KE if the uncertainty in your
mass measurement is 0.3 kg and the uncertainty in your velocity measurement is 0.008 m/s?
(Remember to include the unit in your answer)
\end{question}

\begin{solution}
\examspace*{10em}


\end{solution}


\begin{question}
A pressure transducer is connected to a digital panel meter. The panel meter converts the pressure transducer’s output in volts back to pressure in psi. The manufacturer provides the following information about the panel meter: \\
\begin{tabular}{|cc|}
\hline 
Resolution & 0.1 psi \\ 
Repeatability & 0.1 psi \\ 
Linearity & with 0.1\% of reading \\ 
Drift & less than 0.1 psi /6 months within 32 $^{\circ}$F to 90 $^{\circ}$F range  \\ 
\hline 
\end{tabular} \\

The only information given about the pressure transducer is that it has “an accuracy
of within 0.5 \% of its reading”. Estimate the combined standard uncertainty in a measured pressure at a nominal
value of 100 psi at 70 $^{\circ}$ F. Assume that the transducer’s response is linear with an output of 1 V for every psi of input.
\end{question}

\begin{solution}

\examspace*{10em}

\end{solution}



\begin{question}
Determine the combined standard uncertainty in the density of air,
assuming $\rho = P/RT$. Assume negligible uncertainty in R (R air = 287.04 J/kg·K). Let
T = 24 $^{\circ}$C = 297 $^{\circ}$K and P = 760 mmHg.

\end{question}
\begin{solution}

\examspace*{10em}

\end{solution}

\begin{question}


A group of biomedical engineering students wish to
determine the density of an elliptical cone to be used in the
design of a prosthetic device. They plan to determine the
density from measurements of the cone's mass, length h,
and diameter d, which have instrument resolutions of 0.1
lbm, 0.05 in., and 0.0005 in., respectively. The balance used
to measure the weight has an instrument uncertainty
(accuracy) of 1\%. Each of the different rulers used to
measure the length, and diameter presents an instrument uncertainty (accuracy) of 0.5\%. Nominal
values of the mass, length, and diameter are 4.5 lbm, 6.00 in., and 4.0000 in., respectively.

\begin{enumerate}
\item What are the resolution uncertainties for the measurements of the mass, length, and diameter?
\item Compute the sensitivity coefficients of the density with respect to the mass, length, and diameter at
the nominal values.
\item Estimate the zero-order uncertainty in the determination of the density.
\item Which measurement contributes the most to the zero-order uncertainty?
\item What are the absolute instrument uncertainties for the measurements of the mass, length, and
diameter at the nominal values?
\item Estimate the instrument uncertainty in the determination of the density.
\item Estimate the design-stage uncertainty in the determination of the density.
\end{enumerate}
\end{question}
\begin{solution}
\examspace*{10em}

\end{solution}


\end{document}
