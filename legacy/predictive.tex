
\begin{question}
Use ppm technique, and the following probabilities. Show the different context tables after coding the sequence “AGAAGT”. Assume max context of 2.

\end{question}
\begin{solution}
\begin{enumerate}
\item -1 order context. \\
\begin{tabular}{|c|c|c|}
\hline 
Letter  & Count & CumCount \\ 
\hline 
A & 1 & 1 \\ 
\hline 
C & 1 & 2 \\ 
\hline 
G & 2 & 4 \\ 
\hline 
T & 4 & 8 \\ 
\hline 
\end{tabular} 
\item zero order context. \\
\begin{tabular}{|c|c|c|}
\hline 
Letter  & Count & CumCount \\ 
\hline 
A & 3 & 1 \\ 
\hline 
C & 2 & 2 \\ 
\hline 
T & 1 & 4 \\ 
\hline 
Esc & 1 & 8 \\ 
\hline 
\end{tabular} 
\item first order context. \\
\begin{tabular}{|c|c|c|c|}
\hline 
Context & Letter & Count & CumCount \\ 
\hline 
A & G & 2 & 2 \\ 
\hline 
• & A & 1 & 3 \\ 
\hline 
• & Esc & 1 & 4 \\ 
\hline 
G & A & 1 & 1 \\ 
\hline 
• & T & 1 & 2 \\ 
\hline 
• & Esc & 1 & 3 \\ 
\hline 
\end{tabular} 
\item second order context. \\
\begin{tabular}{|c|c|c|c|}
\hline 
Context & Letter & Count & CumCount \\ 
\hline 
AG & A & 1 & 1 \\ 
\hline 
• & T & 1 & 2 \\ 
\hline 
• & Esc & 1 & 3 \\ 
\hline 
GA & A & 1 & 1 \\ 
\hline 
• & Esc & 1 & 2 \\ 
\hline 
AA & G & 1 & 1 \\ 
\hline 
• & Esc & 1 & 2 \\ 
\hline \hline 

\end{tabular} 


\end{enumerate}
\end{solution}

\begin{question}
The sequence “CAGT” is encoded. Show the generated counts and the cumulative counts.
Modify the above context tables during coding. Use the exclusion principle.
\end{question}
\begin{solution}
\begin{tabular}{|c|c|c|}
\hline 
Letter & Count & CumCount \\ 
\hline 
C & 1 & 2 \\ 
\hline 
A & 3 & 3 \\ 
\hline 
G & 2 & 2 \\ 
\hline 
T & 1 & 2 \\ 
\hline 
\end{tabular}  \\


Updated Tables:
\begin{enumerate}
\item -1 order context. \\
\begin{tabular}{|c|c|c|}
\hline 
Letter  & Count & CumCount \\ 
\hline 
A & 1 & 1 \\ 
\hline 
C & 2 & 3 \\ 
\hline 
G & 4 & 7 \\ 
\hline 
T & 5 & 12 \\ 
\hline 
\end{tabular} 
\item zero order context. \\
\begin{tabular}{|c|c|c|}
\hline 
Letter  & Count & CumCount \\ 
\hline 
A & 4 & 4 \\ 
\hline 
C & 1 & 5 \\ 
\hline 
G & 3 & 8 \\ 
\hline 
T & 2 & 10 \\ 
\hline 
Esc & 1 & 11 \\ 
\hline 
\end{tabular} 
\item first order context. \\
\begin{tabular}{|c|c|c|c|}
\hline 
Context & Letter & Count & CumCount \\ 
\hline 
A & G & 3 & 3 \\ 
\hline 
• & A & 1 & 4 \\ 
\hline 
• & Esc & 1 & 5 \\ 
\hline 
T & C & 1 & 1 \\ 
\hline 
• & Esc & 1 & 2 \\ 
\hline 
C & A & 1 & 1 \\ 
\hline 
• & Esc & 1 & 2 \\ 
\hline 
\end{tabular} 
\item second order context. \\
\begin{tabular}{|c|c|c|c|}
\hline 
Context & Letter & Count & CumCount \\ 
\hline 
AG & A & 1 & 1 \\ 
\hline 
• & T & 2 & 3 \\ 
\hline 
• & Esc & 1 & 1 \\ 
\hline 
GA & A & 1 & 2 \\ 
\hline 
• & Esc & 1 & 3 \\ 
\hline 
AA & G & 1 & 1 \\ 
\hline 
• & Esc & 1 & 2 \\ 
\hline 
GT & C & 1 & 1 \\ 
\hline 
• & Esc & 1 & 2 \\ 
\hline 
TC & A & 1 & 1 \\ 
\hline 
• & Esc & 1 & 2 \\ 
\hline 
CA & G & 1 & 1 \\ 
\hline 
• & Esc & 1 & 2 \\ 
\hline 
\end{tabular} 


\end{enumerate}




\end{solution}


\begin{question}
Using Burrows-Wheeler transform, and the alphabet \{ A, C, T, N, $\Delta$ \}:
Code the sequence \{ACACAT\}. Let row index start at 0.
\end{question}
\begin{solution}
\begin{enumerate}
\item Permutations \\
A C A C A T \\
C A C A T A \\
A C A T A C \\
C A T A C A \\
A T A C A C \\
T A C A C A 
\item Sorted Permutations \\ 
A C A C A T \\
A C A T A C \\
A T A C A C \\
C A C A T A \\
C A T A C A \\
T A C A C A \\ 
\item First row idx=0. Last column=\{TCCAAA\}. \\
Code = Prefix(0):MTF(TCCAAA).
\end{enumerate}
\end{solution}


\begin{question}
Find the Move to Front code for the sequence \{$N N \Delta \Delta \Delta N$\}.
\end{question}
\begin{solution}
\begin{tabular}{|c|c|c|c|c|}
\hline 
0 & 1 & 2 & 3 & 4 \\ 
\hline 
A & C & T & N & $\Delta$ \\ 
\hline 
N & A & C & T & $\Delta$ \\ 
\hline 
$\Delta$ & N & A & C & T \\ 
\hline 
N & $\Delta$ & A & C & T \\ 
\hline 
\end{tabular} \\
Code=\{3,0,4,0,0,1\}
\end{solution}


\begin{question}
Find the inverse Burrows Wheeler transform of \{3, CNACAAA\}. Row numbers are given
at the left most column.
\end{question}
\begin{solution}
\begin{tabular}{|c|c|c|c|c|c|c|c|}
\hline 
0 & A & • & • & • & • & • & C \\ 
\hline 
1 & A & • & • & • & • & • & N \\ 
\hline 
2 & A & • & • & • & • & • & A \\ 
\hline 
3 & A & N & A & C & A & A & C \\ 
\hline 
4 & C & • & • & • & • & • & A \\ 
\hline 
5 & C & • & • & • & • & • & A \\ 
\hline 
6 & N & • & • & • & • & • & A \\ 
\hline 
\end{tabular} \\
Decoded sequence: ANACAAC. 
\end{solution}

\begin{question}
Assume that the underlined part of the sequence “\underline{ACGACGA}CATGCGA” has already been encoded. Encode next 2 phrases using the Associative Coder of Buyanovsky (ACB).
\end{question}
\begin{solution}
\begin{tabular}{|c|c|c|}
\hline 
Unsorted & • & Sorted \\ 
\hline 
A|CGACGA & 1 & A|CGACGA \\ 
\hline 
AC|GACGA & 2 & ACGA|CGA \\ 
\hline 
ACG|ACGA & 3 & AC|GACGA \\ 
\hline 
ACGA|CGA & 4 & ACGAC|GA \\ 
\hline 
ACGAC|GA & 5 & ACG|ACGA \\ 
\hline 
ACGACG|A & 6 & ACGACG|A \\ 
\hline 
\end{tabular} \\
The best context match between 1 and 2. \\
The best match of content: 2 \\
Output (2-1,1,’A’) \\ \\
\begin{tabular}{|c|c|c|}
\hline 
Unsorted & • & Sorted \\ 
\hline 
A|CGACGACA & 1 & A|CGACGACA \\ 
\hline 
AC|GACGACA & 2 & ACGA|CGACA \\ 
\hline 
ACG|ACGACA & 3 & ACGACGA|CA \\ 
\hline 
ACGA|CGACA & 4 & AC|GACGACA \\ 
\hline 
ACGAC|GACA & 5 & ACGAC|GACA \\ 
\hline 
ACGACG|ACA & 6 & ACGACGAC|A \\ 
\hline 
ACGACGA|CA & 7 & ACG|ACGACA \\ 
\hline 
ACGACGAC|A & 8 & ACGACG|ACA \\ 
\hline 
\end{tabular} \\
The best context match between 1 and 3. \\
The best match of content: none \\
Output (0,0,’T’).

\end{solution}

